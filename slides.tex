\documentclass{beamer}
\usepackage{minted}
\usepackage{epigraph}
\usepackage{subfig}
\usepackage{tikz}
\usetheme{focus}

\title{Shor's Algorithm}
\subtitle{Quantum Computer Programming}
\author{Davide Carnemolla}
\titlegraphic{\includegraphics[scale=0.13]{images/logo.pdf}}
\institute{Department of Mathematics and \\ Computer Science \\ \\ University of Catania}
\date{2022/2023}

\begin{document}
    \begin{frame}
        \maketitle
    \end{frame}
    
    \section{Integer Factorization}

    \begin{frame}{The problem}
        \centering
        \includegraphics[height=2cm,keepaspectratio]{images/problem.pdf}
        \vspace{1cm}
        \begin{block}{Definition}
            Given an integer $N$, the goal of the \textit{integer factorization problem} is to find two primes, $p$ and $q$, such that $N = p \cdot q$.
        \end{block}
        \begin{exampleblock}{Example}
            Given $N = 143$ we know that the solution is the pair $(13, 11).$
        \end{exampleblock}
        
    \end{frame}

    \begin{frame}{RSA}
        \centering
        \begin{block}{RSA}
            $(pk,sk) \leftarrow$ \textbf{keygen}(): \hspace{0.05cm} given two primes $p$ and $q$ \\ 
                \hspace{3.7cm} let $n = pq$ \\ 
                \hspace{3.7cm} choose $e,d$ such that $e \cdot d \equiv 1 \; \text{mod} \; \phi(n)$ \\
                \hspace{3.7cm} \textbf{return} $pk = (n,e), \, sk = (n, d)$ \\
            \vspace{0.5cm}
            $c \leftarrow \textbf{enc}_{pk}(m) = m^e \; mod \; n$ \\
            \vspace{0.5cm}
            $m \leftarrow \textbf{dec}_{sk}(m) = c^d \; mod \; n$
        \end{block}
        \begin{alertblock}{RSA's Security}
            The RSA's Problem can be reduced to the Integer Factorization's Problem.
        \end{alertblock}
    \end{frame}

    \section{Solution in the classical model}

    \begin{frame}{General Number Field Sieve}
        
    \end{frame}

    \begin{frame}{General Number Field Sieve: Complexity}
        
    \end{frame}

    \section{Period finding}

    \section{From factoring to period finding}

    \section{Shor's Algorithm}

    \section{Compare between Shor and GNFS}

    \section{Implementation}
    
    
\end{document}
